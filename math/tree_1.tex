\documentclass{article}

\usepackage[utf8]{inputenc}
\usepackage[english, russian]{babel}

\begin{document}
	\paragraph{Деревья, задача 10}
	\hspace{\fill} \newline
	Сведем задачу к графам. Станции - это вершины. Перегоны - неориентированные ребра. Заметим, что получившийся граф - дерево, тк в нем между каждой вершиной ровно один путь(то есть он связен, тк между любой парой вершин есть путь, и не имеет циклов, тк между каждой парой вершин путь ровно один, а при наличии циклов между двумя вершинами, которые принадлежат какому-нибудь циклу, есть как минимум два пути)
	\newline
	a) Найдем лист в нашем дереве(всегда существует, задача 1). Разделим каждое ребро на два ориентированных. Заметим, что количество входящих ребер в вершину равно количеству исходящих. Тогда запустимся из листа и будем последовательно идти по ребрам, выписывать их и зачеркивать, при этом возвращаться к предку по ребру мы будем в последнюю очередь. Мы закончим в этом же листе, потому что у нас после прохождения вершины ее баланс(кол-во исходящих ребер) уменьшался на единицу, при этом мы не пойдем в вершину с нулевым балансом, поскольку это будет означать, что и входящего в нее ребра не существует. Мы выпишем все ребра, потому что для каждой вершины мы выписывали ребро из предка в нее, потом решали задачу для каждого ее сына, после этого выписывали ребро из нее в предка. Всего ребер 2 * (n - 1) = 198.
	\newline
	б) сделаем пункт а), но выпишем все ребра, кроме последних двух. Заметим, что последние два ребра для нас бесполезны. Назовем вершину новой, если мы ее впервые встречаем, когда идем по выписанным ребрам. После того, как мы посетим последнюю "новую" вершину, мы выпишем хотя бы следующие X ребер, где X - расстояние между нашей стартовой вершиной и последней новой. 
\newline
	$X \neq 0$, тк тогда последняя новая вершина = стартовая вершина, а стартовая вершина была первой новой вершиной. 
\newline
	$ X \neq 1$, тк стартовая вершина была листом, соответственно у нее был единственный сосед на расстоянии 1, который был второй новой вершиной.
	\newline
	Тогда $X \geq 2$, поэтому мы можем выписать 198 - 2 = 196 ребер
\end{document}