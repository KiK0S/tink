\documentclass{article}

\usepackage[english, russian]{babel}
\usepackage[utf8]{inputenc}

\begin{document}
	\paragraph{Серия 6. Задача 9}
	\hspace{\fill}
	\newline
	Для решения возьмем произвольную последовательность из m мальчиков и d девочек, после чего рассмотрим все позиции мальчиков. Для каждой из них справедливы следующие два утверждения:
	\begin{enumerate}
		\item Справа от этого мальчика стоит k девочек, $0 \le k \le d$
		\item Слева от этого мальчика стоит d - k девочек
	\end{enumerate}
	Тогда число, произнесенное мальчиком, добавит в итоговую сумму d - k. При этом k девочек, стоящих справа, подсчитают этого мальчика, тем самым добавив в итоговую сумму k. Тогда каждый из m мальчиков добавляет (k + d - k) = d в сумму.
Заметим, что больше ничего подсчитывать уже не нужно - мы посчитали все пары (девочка, мальчик), когда сказали, что каждый мальчик добавит d - k; все пары (мальчик, девочка) мы посчитали, когда сказали, что k девочек посчитают конкретного мальчика.
Ответ - m * d.
\end{document}